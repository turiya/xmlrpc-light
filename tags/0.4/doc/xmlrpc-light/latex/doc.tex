\documentclass[11pt]{article} 
\usepackage[latin1]{inputenc} 
\usepackage[T1]{fontenc} 
\usepackage{fullpage} 
\usepackage{url} 
\usepackage{ocamldoc}
\begin{document}
\tableofcontents
\section{Module {\tt{XmlRpc}} : XmlRpc Light.}
\label{module:XmlRpc}\index{XmlRpc@\verb`XmlRpc`}



    XmlRpc Light is a minimal XmlRpc library based on Xml Light and Ocamlnet.


    It provides a type for values, a client class with a simple calling
    interface, and low-level tools that can be used to implement a server.


    {\it (c) 2007 Dave Benjamin}



\ocamldocvspace{0.5cm}



High-level interface



Example: \begin{ocamldoccode}

    let rpc = new XmlRpc.client "http://localhost:8000" in
    let result = rpc#call "echo" [`String "hello!"] in
    print_endline (XmlRpc.dump result) 
\end{ocamldoccode}




\label{exception:XmlRpc.Error}\begin{ocamldoccode}
exception Error of (int * string)
\end{ocamldoccode}
\index{Error@\verb`Error`}
\begin{ocamldocdescription}
Raised for all errors including XmlRpc faults (code, string).


\end{ocamldocdescription}




\label{type:XmlRpc.value}\begin{ocamldoccode}
type value = [ `Array of value list
  | `Binary of string
  | `Boolean of bool
  | `DateTime of int * int * int * int * int * int * int
  | `Double of float
  | `Int of int
  | `String of string
  | `Struct of (string * value) list ] 
\end{ocamldoccode}
\index{value@\verb`value`}
\begin{ocamldocdescription}
Polymorphic variant type for XmlRpc values:\begin{itemize}
\item {\tt{`Array}}: An ordered list of values
\item {\tt{`Binary}}: A string containing binary data
\item {\tt{`Boolean}}: A boolean
\item {\tt{`DateTime}}: A date-time value
      (year, month, day, hour, minute, second, timezone offset in minutes)
\item {\tt{`Double}}: A floating-point value
\item {\tt{`Int}}: An integer
\item {\tt{`String}}: A string
\item {\tt{`Struct}}: An association list of (name, value) pairs
\end{itemize}

    Note that base64-encoding of {\tt{`Binary}} values is done automatically.
    You do not need to do the encoding yourself.


\end{ocamldocdescription}




\begin{ocamldoccode}
{\tt{class client : }}{\tt{string -> }}\end{ocamldoccode}
\label{class:XmlRpc.client}\index{client@\verb`client`}

\begin{ocamldocobjectend}


\label{val:XmlRpc.client.url}\begin{ocamldoccode}
val url : string
\end{ocamldoccode}
\index{url@\verb`url`}
\begin{ocamldocdescription}
Url of the remote XmlRpc server.


\end{ocamldocdescription}


\label{val:XmlRpc.client.useragent}\begin{ocamldoccode}
val mutable useragent : string
\end{ocamldoccode}
\index{useragent@\verb`useragent`}
\begin{ocamldocdescription}
User-agent to send in request headers.


\end{ocamldocdescription}


\label{val:XmlRpc.client.debug}\begin{ocamldoccode}
val mutable debug : bool
\end{ocamldoccode}
\index{debug@\verb`debug`}
\begin{ocamldocdescription}
If true, Xml messages will be printed to standard output.


\end{ocamldocdescription}


\label{method:XmlRpc.client.url}\begin{ocamldoccode}
method url : string
\end{ocamldoccode}
\index{url@\verb`url`}
\begin{ocamldocdescription}
Gets {\tt{url}}.


\end{ocamldocdescription}


\label{method:XmlRpc.client.useragent}\begin{ocamldoccode}
method useragent : string
\end{ocamldoccode}
\index{useragent@\verb`useragent`}
\begin{ocamldocdescription}
Gets {\tt{useragent}}.


\end{ocamldocdescription}


\label{method:XmlRpc.client.set-underscoreuseragent}\begin{ocamldoccode}
method set_useragent : string -> unit
\end{ocamldoccode}
\index{set-underscoreuseragent@\verb`set_useragent`}
\begin{ocamldocdescription}
Sets {\tt{useragent}}.


\end{ocamldocdescription}


\label{method:XmlRpc.client.debug}\begin{ocamldoccode}
method debug : bool
\end{ocamldoccode}
\index{debug@\verb`debug`}
\begin{ocamldocdescription}
Gets {\tt{debug}}.


\end{ocamldocdescription}


\label{method:XmlRpc.client.set-underscoredebug}\begin{ocamldoccode}
method set_debug : bool -> unit
\end{ocamldoccode}
\index{set-underscoredebug@\verb`set_debug`}
\begin{ocamldocdescription}
Sets {\tt{debug}}.


\end{ocamldocdescription}


\label{method:XmlRpc.client.set-underscorebase64-underscoreencode}\begin{ocamldoccode}
method set_base64_encode : (string -> string) -> unit
\end{ocamldoccode}
\index{set-underscorebase64-underscoreencode@\verb`set_base64_encode`}
\begin{ocamldocdescription}
Sets an alternate Base-64 binary encoding function.


\end{ocamldocdescription}


\label{method:XmlRpc.client.set-underscorebase64-underscoredecode}\begin{ocamldoccode}
method set_base64_decode : (string -> string) -> unit
\end{ocamldoccode}
\index{set-underscorebase64-underscoredecode@\verb`set_base64_decode`}
\begin{ocamldocdescription}
Sets an alternate Base-64 binary decoding function.


\end{ocamldocdescription}


\label{method:XmlRpc.client.set-underscoredatetime-underscoreencode}\begin{ocamldoccode}
method set_datetime_encode :
  (int * int * int * int * int * int * int -> string) -> unit
\end{ocamldoccode}
\index{set-underscoredatetime-underscoreencode@\verb`set_datetime_encode`}
\begin{ocamldocdescription}
Sets an alternate ISO-8601 date/time encoding function.


\end{ocamldocdescription}


\label{method:XmlRpc.client.set-underscoredatetime-underscoredecode}\begin{ocamldoccode}
method set_datetime_decode :
  (string -> int * int * int * int * int * int * int) -> unit
\end{ocamldoccode}
\index{set-underscoredatetime-underscoredecode@\verb`set_datetime_decode`}
\begin{ocamldocdescription}
Sets an alternate ISO-8601 date/time decoding function.


\end{ocamldocdescription}


\label{method:XmlRpc.client.call}\begin{ocamldoccode}
method call : string -> XmlRpc.value list -> XmlRpc.value
\end{ocamldoccode}
\index{call@\verb`call`}
\begin{ocamldocdescription}
{\tt{call name params}} invokes an XmlRpc method and returns the result,
      or raises {\tt{XmlRpc.Error}}[\ref{exception:XmlRpc.Error}] on error.


\end{ocamldocdescription}
\end{ocamldocobjectend}


\begin{ocamldocdescription}
Class for XmlRpc clients. Takes a single argument, the Url.


\end{ocamldocdescription}




Utility functions



\label{val:XmlRpc.dump}\begin{ocamldoccode}
val dump : value -> string
\end{ocamldoccode}
\index{dump@\verb`dump`}
\begin{ocamldocdescription}
Converts an XmlRpc value to a human-readable string.


\end{ocamldocdescription}




\label{val:XmlRpc.iso8601-underscoreof-underscoredatetime}\begin{ocamldoccode}
val iso8601_of_datetime : int * int * int * int * int * int * int -> string
\end{ocamldoccode}
\index{iso8601-underscoreof-underscoredatetime@\verb`iso8601_of_datetime`}
\begin{ocamldocdescription}
Converts a date/time tuple to an ISO-8601 string.


\end{ocamldocdescription}




\label{val:XmlRpc.datetime-underscoreof-underscoreiso8601}\begin{ocamldoccode}
val datetime_of_iso8601 : string -> int * int * int * int * int * int * int
\end{ocamldoccode}
\index{datetime-underscoreof-underscoreiso8601@\verb`datetime_of_iso8601`}
\begin{ocamldocdescription}
Converts an ISO-8601 string to a date/time tuple.


\end{ocamldocdescription}




Low-level interface



\label{type:XmlRpc.message}\begin{ocamldoccode}
type message =
  | MethodCall of (string * value list)
  | MethodResponse of value
  | Fault of (int * string)
\end{ocamldoccode}
\index{message@\verb`message`}
\begin{ocamldocdescription}
Type for XmlRpc messages.


\end{ocamldocdescription}




\label{val:XmlRpc.message-underscoreof-underscorexml-underscoreelement}\begin{ocamldoccode}
val message_of_xml_element :
  ?base64_decode:(string -> string) ->
  ?datetime_decode:(string -> int * int * int * int * int * int * int) ->
  Xml.xml -> message
\end{ocamldoccode}
\index{message-underscoreof-underscorexml-underscoreelement@\verb`message_of_xml_element`}
\begin{ocamldocdescription}
Converts an Xml Light element to an XmlRpc message.


\end{ocamldocdescription}




\label{val:XmlRpc.xml-underscoreelement-underscoreof-underscoremessage}\begin{ocamldoccode}
val xml_element_of_message :
  ?base64_encode:(string -> string) ->
  ?datetime_encode:(int * int * int * int * int * int * int -> string) ->
  message -> Xml.xml
\end{ocamldoccode}
\index{xml-underscoreelement-underscoreof-underscoremessage@\verb`xml_element_of_message`}
\begin{ocamldocdescription}
Converts an XmlRpc message to an Xml Light element.


\end{ocamldocdescription}




\label{val:XmlRpc.value-underscoreof-underscorexml-underscoreelement}\begin{ocamldoccode}
val value_of_xml_element :
  ?base64_decode:(string -> string) ->
  ?datetime_decode:(string -> int * int * int * int * int * int * int) ->
  Xml.xml -> value
\end{ocamldoccode}
\index{value-underscoreof-underscorexml-underscoreelement@\verb`value_of_xml_element`}
\begin{ocamldocdescription}
Converts an Xml Light element to an XmlRpc value.


\end{ocamldocdescription}




\label{val:XmlRpc.xml-underscoreelement-underscoreof-underscorevalue}\begin{ocamldoccode}
val xml_element_of_value :
  ?base64_encode:(string -> string) ->
  ?datetime_encode:(int * int * int * int * int * int * int -> string) ->
  value -> Xml.xml
\end{ocamldoccode}
\index{xml-underscoreelement-underscoreof-underscorevalue@\verb`xml_element_of_value`}
\begin{ocamldocdescription}
Converts an XmlRpc value to an Xml Light element.


\end{ocamldocdescription}




Server tools



\label{val:XmlRpc.serve}\begin{ocamldoccode}
val serve :
  ?base64_encode:(string -> string) ->
  ?base64_decode:(string -> string) ->
  ?datetime_encode:(int * int * int * int * int * int * int -> string) ->
  ?datetime_decode:(string -> int * int * int * int * int * int * int) ->
  ?error_handler:(exn -> message) ->
  (string -> value list -> value) -> string -> string
\end{ocamldoccode}
\index{serve@\verb`serve`}
\begin{ocamldocdescription}
Creates a function from string (Xml representing a {\tt{MethodCall}}) to
    string (Xml representing a {\tt{MethodResult}} or {\tt{Fault}}) given a function
    of the form: ({\tt{name}} $\rightarrow$ {\tt{params}} $\rightarrow$ {\tt{result}}), where {\tt{name}} is the
    name of the method, {\tt{params}} is a list of parameter values, and
    {\tt{result}} is the result value.


    This function can be used to build many different kinds of XmlRpc
    servers since it makes no assumptions about the network library
    or other communications method used.


    If an exception other than {\tt{XmlRpc.Error}}[\ref{exception:XmlRpc.Error}] occurs, the exception is
    passed to {\tt{error\_handler}}. If {\tt{error\_handler}} returns a message,
    the message will be used as the result. If an {\tt{XmlRpc.Error}}[\ref{exception:XmlRpc.Error}] is
    raised by either the main function or {\tt{error\_handler}}, it will be
    converted to an XmlRpc {\tt{Fault}}. Any other exception raised by
    {\tt{error\_handler}} is allowed to escape.


    For a full-featured, easy-to-use, network-capable server implementation,
    see the {\tt{XmlRpcServer}}[\ref{module:XmlRpcServer}] module.


\end{ocamldocdescription}




\label{val:XmlRpc.default-underscoreerror-underscorehandler}\begin{ocamldoccode}
val default_error_handler : exn -> message
\end{ocamldoccode}
\index{default-underscoreerror-underscorehandler@\verb`default_error_handler`}
\begin{ocamldocdescription}
The default error handler for {\tt{serve}}.


    This error handler catches all exceptions and converts them into
    faults by wrapping them in {\tt{XmlRpc.Error}}.


\end{ocamldocdescription}




\label{val:XmlRpc.quiet-underscoreerror-underscorehandler}\begin{ocamldoccode}
val quiet_error_handler : exn -> message
\end{ocamldoccode}
\index{quiet-underscoreerror-underscorehandler@\verb`quiet_error_handler`}
\begin{ocamldocdescription}
A "quiet" error handler for {\tt{serve}}.


    This error handler simply re-raises the exception. Use this if you
    want exceptions to remain unhandled so that they will escape to the
    error log. The client will receive a generic "transport error",
    which is more secure since it does not reveal any information about
    the specific exception that occurred.


\end{ocamldocdescription}


\section{Module {\tt{XmlRpcServer}} : XmlRpc Light server.}
\label{module:XmlRpcServer}\index{XmlRpcServer@\verb`XmlRpcServer`}




\ocamldocvspace{0.5cm}



Example: \begin{ocamldoccode}

    let server = new XmlRpcServer.cgi () in
    server#register "demo.sayHello"
      (fun _ -> `String "Hello!");
    server#run () 
\end{ocamldoccode}



    By inheriting from {\tt{XmlRpcServer.base}}[\ref{class:XmlRpcServer.base}], all servers provide
    the following introspection functions by default: {\tt{system.listMethods}},
    {\tt{system.getCapabilities}}. To prevent their use, use {\tt{server\verb`#`unregister}}.



Base classes



\begin{ocamldoccode}
{\tt{class virtual base : }}\end{ocamldoccode}
\label{class:XmlRpcServer.base}\index{base@\verb`base`}

\begin{ocamldocobjectend}


\label{val:XmlRpcServer.base.methods}\begin{ocamldoccode}
val methods : (string, XmlRpc.value list -> XmlRpc.value) Hashtbl.t
\end{ocamldoccode}
\index{methods@\verb`methods`}
\begin{ocamldocdescription}
Hashtable mapping method names to implementation functions.


\end{ocamldocdescription}


\label{method:XmlRpcServer.base.set-underscorebase64-underscoreencode}\begin{ocamldoccode}
method set_base64_encode : (string -> string) -> unit
\end{ocamldoccode}
\index{set-underscorebase64-underscoreencode@\verb`set_base64_encode`}
\begin{ocamldocdescription}
Sets an alternate Base-64 binary encoding function.


\end{ocamldocdescription}


\label{method:XmlRpcServer.base.set-underscorebase64-underscoredecode}\begin{ocamldoccode}
method set_base64_decode : (string -> string) -> unit
\end{ocamldoccode}
\index{set-underscorebase64-underscoredecode@\verb`set_base64_decode`}
\begin{ocamldocdescription}
Sets an alternate Base-64 binary decoding function.


\end{ocamldocdescription}


\label{method:XmlRpcServer.base.set-underscoredatetime-underscoreencode}\begin{ocamldoccode}
method set_datetime_encode :
  (int * int * int * int * int * int * int -> string) -> unit
\end{ocamldoccode}
\index{set-underscoredatetime-underscoreencode@\verb`set_datetime_encode`}
\begin{ocamldocdescription}
Sets an alternate ISO-8601 date/time encoding function.


\end{ocamldocdescription}


\label{method:XmlRpcServer.base.set-underscoredatetime-underscoredecode}\begin{ocamldoccode}
method set_datetime_decode :
  (string -> int * int * int * int * int * int * int) -> unit
\end{ocamldoccode}
\index{set-underscoredatetime-underscoredecode@\verb`set_datetime_decode`}
\begin{ocamldocdescription}
Sets an alternate ISO-8601 date/time decoding function.


\end{ocamldocdescription}


\label{method:XmlRpcServer.base.set-underscoreerror-underscorehandler}\begin{ocamldoccode}
method set_error_handler : (exn -> XmlRpc.message) -> unit
\end{ocamldoccode}
\index{set-underscoreerror-underscorehandler@\verb`set_error_handler`}
\begin{ocamldocdescription}
Sets an alternate handler for unhandled exceptions.
      See {\tt{XmlRpc.default\_error\_handler}}[\ref{val:XmlRpc.default-underscoreerror-underscorehandler}] and
      {\tt{XmlRpc.quiet\_error\_handler}}[\ref{val:XmlRpc.quiet-underscoreerror-underscorehandler}] for examples.


\end{ocamldocdescription}


\label{method:XmlRpcServer.base.register}\begin{ocamldoccode}
method register : string -> (XmlRpc.value list -> XmlRpc.value) -> unit
\end{ocamldoccode}
\index{register@\verb`register`}
\begin{ocamldocdescription}
Registers a method with the server.


\end{ocamldocdescription}


\label{method:XmlRpcServer.base.unregister}\begin{ocamldoccode}
method unregister : string -> unit
\end{ocamldoccode}
\index{unregister@\verb`unregister`}
\begin{ocamldocdescription}
Removes a method from the server.


\end{ocamldocdescription}


\label{method:XmlRpcServer.base.run}\begin{ocamldoccode}
method virtual run : unit -> unit
\end{ocamldoccode}
\index{run@\verb`run`}
\begin{ocamldocdescription}
Starts the main server process.


\end{ocamldocdescription}
\end{ocamldocobjectend}


\begin{ocamldocdescription}
Abstract base class for XmlRpc servers.


\end{ocamldocdescription}




\begin{ocamldoccode}
{\tt{class type server = }}\end{ocamldoccode}
\label{classtype:XmlRpcServer.server}\index{server@\verb`server`}

\begin{ocamldocobjectend}


{\tt{inherit XmlRpcServer.base}} [\ref{class:XmlRpcServer.base}]

\label{method:XmlRpcServer.server.run}\begin{ocamldoccode}
method run : unit -> unit
\end{ocamldoccode}
\index{run@\verb`run`}
\begin{ocamldocdescription}
Starts the main server process.


\end{ocamldocdescription}
\end{ocamldocobjectend}


\begin{ocamldocdescription}
Type of concrete XmlRpc server classes.


\end{ocamldocdescription}




Server implementations



\begin{ocamldoccode}
{\tt{class cgi : }}{\tt{unit -> }}{\tt{server}}\end{ocamldoccode}
\label{class:XmlRpcServer.cgi}\index{cgi@\verb`cgi`}



\begin{ocamldocdescription}
CGI XmlRpc server based on Netcgi2.


\end{ocamldocdescription}




\begin{ocamldoccode}
{\tt{class netplex : }}{\tt{?parallelizer:Netplex\_types.parallelizer -> ?handler:string -> unit -> }}{\tt{server}}\end{ocamldoccode}
\label{class:XmlRpcServer.netplex}\index{netplex@\verb`netplex`}



\begin{ocamldocdescription}
Stand-alone XmlRpc server based on Netplex.


\end{ocamldocdescription}




Utility functions



\label{val:XmlRpcServer.invalid-underscoremethod}\begin{ocamldoccode}
val invalid_method : string -> 'a
\end{ocamldoccode}
\index{invalid-underscoremethod@\verb`invalid_method`}
\begin{ocamldocdescription}
Raise an {\tt{XmlRpc.Error}}[\ref{exception:XmlRpc.Error}] indicating a method name not found.


\end{ocamldocdescription}




\label{val:XmlRpcServer.invalid-underscoreparams}\begin{ocamldoccode}
val invalid_params : unit -> 'a
\end{ocamldoccode}
\index{invalid-underscoreparams@\verb`invalid_params`}
\begin{ocamldocdescription}
Raise an {\tt{XmlRpc.Error}}[\ref{exception:XmlRpc.Error}] indicating invalid method parameters.


\end{ocamldocdescription}


\section{Module {\tt{XmlRpcBase64}} : Base64 codec.}
\label{module:XmlRpcBase64}\index{XmlRpcBase64@\verb`XmlRpcBase64`}



	8-bit characters are encoded into 6-bit ones using ASCII lookup tables.
	Default tables maps 0..63 values on characters A-Z, a-z, 0-9, '+' and '/'
	(in that order).



\ocamldocvspace{0.5cm}



\label{exception:XmlRpcBase64.Invalid-underscorechar}\begin{ocamldoccode}
exception Invalid_char
\end{ocamldoccode}
\index{Invalid-underscorechar@\verb`Invalid_char`}
\begin{ocamldocdescription}
This exception is raised when reading an invalid character
	from a base64 input.


\end{ocamldocdescription}




\label{exception:XmlRpcBase64.Invalid-underscoretable}\begin{ocamldoccode}
exception Invalid_table
\end{ocamldoccode}
\index{Invalid-underscoretable@\verb`Invalid_table`}
\begin{ocamldocdescription}
This exception is raised if the encoding or decoding table
	size is not correct.


\end{ocamldocdescription}




\label{type:XmlRpcBase64.encoding-underscoretable}\begin{ocamldoccode}
type encoding_table = char array 
\end{ocamldoccode}
\index{encoding-underscoretable@\verb`encoding_table`}
\begin{ocamldocdescription}
An encoding table maps integers 0..63 to the corresponding char.


\end{ocamldocdescription}




\label{type:XmlRpcBase64.decoding-underscoretable}\begin{ocamldoccode}
type decoding_table = int array 
\end{ocamldoccode}
\index{decoding-underscoretable@\verb`decoding_table`}
\begin{ocamldocdescription}
A decoding table maps chars 0..255 to the corresponding 0..63 value
 or -1 if the char is not accepted.


\end{ocamldocdescription}




\label{val:XmlRpcBase64.str-underscoreencode}\begin{ocamldoccode}
val str_encode : ?tbl:encoding_table -> string -> string
\end{ocamldoccode}
\index{str-underscoreencode@\verb`str_encode`}
\begin{ocamldocdescription}
Encode a string into Base64.


\end{ocamldocdescription}




\label{val:XmlRpcBase64.str-underscoredecode}\begin{ocamldoccode}
val str_decode : ?tbl:decoding_table -> string -> string
\end{ocamldoccode}
\index{str-underscoredecode@\verb`str_decode`}
\begin{ocamldocdescription}
Decode a string encoded into Base64, raise {\tt{Invalid\_char}} if a
	character in the input string is not a valid one.


\end{ocamldocdescription}




\label{val:XmlRpcBase64.encode}\begin{ocamldoccode}
val encode : ?tbl:encoding_table -> char Stream.t -> char Stream.t
\end{ocamldoccode}
\index{encode@\verb`encode`}
\begin{ocamldocdescription}
Generic base64 encoding over a character stream.


\end{ocamldocdescription}




\label{val:XmlRpcBase64.decode}\begin{ocamldoccode}
val decode : ?tbl:decoding_table -> char Stream.t -> char Stream.t
\end{ocamldoccode}
\index{decode@\verb`decode`}
\begin{ocamldocdescription}
Generic base64 decoding over a character stream.


\end{ocamldocdescription}




\label{val:XmlRpcBase64.make-underscoredecoding-underscoretable}\begin{ocamldoccode}
val make_decoding_table : encoding_table -> decoding_table
\end{ocamldoccode}
\index{make-underscoredecoding-underscoretable@\verb`make_decoding_table`}
\begin{ocamldocdescription}
Create a valid decoding table from an encoding one.


\end{ocamldocdescription}


\end{document}